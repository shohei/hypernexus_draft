\documentclass[]{article}

%%%%%%%%%%%%%%%%%%%%%%%%%%
\usepackage[printwatermark]{xwatermark}
\usepackage{xcolor}
\newwatermark[allpages,color=gray!10, angle=45,scale=5,xpos=0,ypos=0]{DRAFT}
%%%%%%%%%%%%%%%%%%%%%%%%%%

\begin{document}
\author{Shohei Aoki}
\title{Consideration of Interfacing Local Knowledge}
\date{First draft, 25 Jul 2015}  
\maketitle

\section{Significance of local knowledge}
How to utilize local knowledge, skill, and techniques that inherent in the local designer is one of the issue to achieve the design of machines that considers local context. 
An example of the products that utilises such local knowledge is {\it rot-itaen}, which is a locally made vehicle in the local garage in Thailand.
The significant feature of the vehicle is its transformable function.
Normally, it is used as a truck to carry both human and load. It also works as an irrigation pump by removing its engine from body.  
The idea itself can be seen in other places, since it is generally regarded as module-based architecture of machines. 
However, The author believes that the concept of the machine is defined {\t after} developing the machine. 
In this perspective, this design process takes a bottom-up approach while development.
In the design process, local knowledge ranging from various domain must have been accumulated.
In this article, the author clarifies how to generate synergy between designers who have own different local knowledge. 

\section{Hyper nexus}
The author names such a bridge as {\it hyper nexus}, which is named after the hyper link in the Internet technology.

\section{Format of describing hyper nexus}

\section{Knowledge base}

\section{Experiment}


\section{Result}


\begin{thebibliography}{99}
\item S.Aoki, Design support for creative problem solving in developing countries, PhD thesis at the University of Tokyo, 2015
\end{thebibliography}

\end{document}
